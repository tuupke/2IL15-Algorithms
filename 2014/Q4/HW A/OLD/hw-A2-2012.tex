\documentclass{article}

\usepackage{a4wide,amsmath,amssymb,fancyhdr,graphicx,tabularx,xspace,algorithmic}

\pagestyle{fancy}
\chead{}
\lhead{TU Eindhoven}
\rhead{Algorithms (2IL15) --- Homework Exercises}
\cfoot{\thepage}
\lfoot{}
\rfoot{}

%to include IPE/pdf correctly
\expandafter\ifx\csname pdfoptionalwaysusepdfpagebox\endcsname\relax\else
\pdfoptionalwaysusepdfpagebox5
\fi

%comments command
\newcommand{\complain}[1]{\begin{quotation} {\sf *** #1 ***} \end{quotation}}


%
% Macros
%


\newcommand{\Reals}{{\Bbb R}}
\newcommand{\Nats}{{\Bbb N}}
\newcommand{\Ints}{{\Bbb Z}}


\newcommand{\C}{\ensuremath{\mathcal{C}}}
\newcommand{\E}{\ensuremath{\mathcal{E}}}
\newcommand{\F}{\ensuremath{\mathcal{F}}}
\newcommand{\G}{\ensuremath{\mathcal{G}}}
\newcommand{\U}{\ensuremath{\mathcal{U}}}

\newcommand{\tree}{\ensuremath{\mathcal{T}}}
\newcommand{\node}{\nu}
\newcommand{\lchild}{\mathrm{lc}}
\newcommand{\rchild}{\mathrm{rc}}
\newcommand{\size}{\mathit{size}}
\newcommand{\cost}{\mathit{cost}}
\newcommand{\leaf}{\mu}
\newcommand{\mylist}{{\cal L}}
\newcommand{\myroot}{\mathit{root}}
\newcommand{\key}{\mathit{key}}
\newcommand{\bd}{\partial}





\newcommand{\eps}{\varepsilon}
\newcommand{\ol}{\overline}
\renewcommand{\leq}{\leqslant}
\renewcommand{\geq}{\geqslant}



\newcommand{\pr}[1]{\Pr[#1]}
\DeclareMathOperator{\expectation}{E}
\newcommand{\expt}[1]{\expectation[#1]}
\newcommand{\events}[1]{\mbox{Events}(#1)}
\newcommand{\rank}{\mathit{rank}}
\newcommand{\result}{\mathit{result}}
\newcommand{\piv}{\mathrm{piv}}
\newcommand{\myexp}{\mathrm{exp}}
\newcommand{\best}{\mathrm{best}}
\newcommand{\worst}{\mathrm{worst}}
\newcommand{\dest}{\mathit{dest}}
\newcommand{\dist}{\mathit{distance}}
\newcommand{\weight}{\mathit{weight}}
\newcommand{\alg}{{\sc Alg}\xspace}
\newcommand{\myleft}{\mathit{left}}
\newcommand{\myright}{\mathit{right}}

\newcommand{\start}{\mathit{start}}
\newcommand{\myend}{\mathit{end}}
\newcommand{\free}{\mathit{free}}

\newcommand{\etal}{{\emph{et al.}\xspace}}
%%
% Theorem-Like Environments
%
\newtheorem{defin}{Definition}
  \newenvironment{mydefinition}{\begin{defin} \sl}{\end{defin}}
\newtheorem{theo}[defin]{Theorem}
  \newenvironment{mytheorem}{\begin{theo} \sl}{\end{theo}}
\newtheorem{lem}[defin]{Lemma}
  \newenvironment{mylemma}{\begin{lem} \sl}{\end{lem}}
\newtheorem{propo}[defin]{Proposition}
  \newenvironment{myproposition}{\begin{propo} \sl}{\end{propo}}
\newtheorem{coro}[defin]{Corollary}
  \newenvironment{corollary}{\begin{coro} \sl}{\end{coro}}
%\newtheorem{obse}[defin]{Observation}
%  \newenvironment{observation}{\begin{obse} \sl}{\end{obse}}
%\newtheorem{rem}[defin]{Remark}
%  \newenvironment{remark}{\begin{rem} \rm}{\end{rem}}

\newenvironment{myproof}{\emph{Proof.}}{\hfill $\Box$ \medskip\\}

% list environments with less spacing
\newcommand{\BeginMyItemize}{\begin{itemize}\setlength{\itemsep}{-\parskip}}
\newcommand{\EndMyItemize}{\end{itemize}}
\newcommand{\myitemize}[1]{\BeginMyItemize #1 \EndMyItemize}

\newcounter{rcounter}
\newenvironment{rlist}%
{\begin{list}{A.2-\arabic{rcounter}}{\usecounter{rcounter}}}{\end{list}}
\newcounter{rcountermem}

%%%%%%%%%%%%%%%%%%%%%%%%%%%%%%%%%%%%%%%%%%%%%%%%%%%%%%%%%%%%%%%%%%%%%%%%%%%

\title{Algorithms (2IL15): Homework Set A.2}
\author{Mart Pluijmaekers, 0753117}
\date{9-5-2012}





%%%%%%%%%%%%%%%%%%%%%%%%%%%%%%%%%%%%%%%%%%%%%%%%%%%%%%%%%%%%%%%%%%%%%%%%%%%

\begin{document}

\section*{Algorithms (2IL15): Homework Set A.2}

\subsection*{Exercises for Lecture 3}
\begin{rlist}
\item Answers:
\begin{enumerate}
      \item[(i)] I would use the greedy choice since the excercise states that the algorithm is correct and a correct greedy algorithm is always quicker then a dynamic programming algorithm.
      \item[(ii)] This is not correct, this can be shown by the following counterexample.\\ Let A be a $2*2$ matrix, B be a $2*10$ matrix and C be a $10*3$ matrix. Then there are 2 possibilities for calculating the total multiplication, namely: $(A*B)*C$ or $A*(B*C)$. \\ Since the initial cost of the first multiplication for the first option would a cost of $2*2*10=40$ and the first multiplication for the second choice would be $2*10*3=60$. So the greedy choice would go for the first choice, but when the second step is calculated the minimal cost picture changes. The first option would become $2*2*10+2*10*3=100$ while the cost for the second option would become $2*10*2+2*2*3=72$. Since the choice of the greedy algorithm was incorrect it shows that the greedy algorithm does not always give the most optimal solution.
      \end{enumerate}

\item ($\frac{1}{2}+1+1+\frac{1}{2}+1$ points)
      
      \begin{enumerate}
      \item[(i)] Given $i,j$ with $1\leq i \le j \leq n$ compute a cluster tree with minimal cost for $[x_i ... x_j]$.
$m[i,j]=$minimal cost to compute a cluster tree for $[x_i ... x_j]$.
      \item[(ii)] \[
m[i,j]=
\begin{cases}
0, &\text{if $i=j$;}\\
min_{i\leq k\textless j} \{m[i,k]+m[k+1,j] +X_{i,j}, &\text{if $i<j$;} 
\end{cases}
\]

Proof:\\
If $i=j$ trivial case.

For the other case we consider a optimal solution OPT for computing a clustertree on $\{x_i,..,x_k\}$ and let j be the index such that OPT is of the form: $\{x_i,..,x_j\} \cup {x_j,..,x_k}$. Then we can state that m[$i,k$]=the cost in OPT for $\{x_i,..,x_j\} + $ the cost in OPT for $\{x_{j+1},..,x_k\} +$ cost$(X_{i,k}) \ge m[i,j]+m[j+1,k]+\text{cost}(X_{i,k}) \ge \text{min}_{i\le l < k}(m[i,l]+m[l+1,k]+\text{cost}(X_{i,k})$ such that $m[i,k]\le \text{min}_{i\le l < k}(m[i,l]+m[l+1,k]+\text{cost}(X_{i,k}))$.
      \item[(iii)]

\begin{algorithmic}[1]
\STATE{CalcCost(Cost}
\FOR{$i\gets 1 \text{to} n$}
\STATE c[$i,i] \gets 0$
\ENDFOR
\FOR{$i\gets n-1 to 1$}
\FOR{$j\gets i+1 to n $}
\STATE{$c[i,j] \gets \text{min}_{i\le k<j}(c[i,k],c[k+1,j)+\text{Cost}[i,j]$}
\ENDFOR
\ENDFOR
\RETURN{c[1,n]}
\end{algorithmic}



      \item[(iv)] The first loop is a single loop so it takes $O(n)$, lines 5, 6, 8 and 9 define 2 nested loops so this part takes $O(n^2)$ but since line 7 is actually a loop itself which iterates over n the actual running time is $O(n^3)$
      \item[(v)] 
\begin{algorithmic}[1]
\STATE{CalcCostExt(Cost}
\FOR{$i\gets 1 \text{to} n$}
\STATE c[$i,i] \gets 0$
\ENDFOR
\FOR{$i\gets n-1 to 1$}
\FOR{$j\gets i+1 to n $}
\STATE{$c[i,j]\gets \infty$}
\FOR{$k \gets i to j-1$}
\STATE{$c \gets\text{min}_{i\le k<j}(c[i,k],c[k+1,j)+\text{Cost}[i,j]$}
\IF{$c < c[i,j]$}
\STATE{$c[i,j] \gets c$}
\STATE{tree$[i,j] \gets k$}
\ENDIF
\ENDFOR
\ENDFOR
\ENDFOR
\RETURN{c[1,n]}
\end{algorithmic}
\bigskip
\begin{algorithmic}[1]
\STATE{PrintClTree(i,j)}
\IF{i=j}
\RETURN{$x_i$}
\ELSE
\RETURN{X$_{ij}$ with left child PrintClTree($i$,tree[$i,j])$ and right child PrintClTree(s[i,j]+1,j)}
\ENDIF
\end{algorithmic}
      \end{enumerate}
\setcounter{rcountermem}{\value{rcounter}}
\end{rlist}
\subsection*{Exercises for Lecture 4}
\begin{rlist}
\setcounter{rcounter}{\value{rcountermem}}
\item ($\frac{1}{2}+1+\frac{1}{2}$ points) 
      \begin{enumerate}
      \item[(i)] 
Given $i,j$ with $0\leq i \leq n+1$ compute a set of petrol stations with minimal waiting time for $[x_i ... x_{n+1}]$. \\
$m[i,j]=$minimal waiting time on petrol stations to travel  for $[x_i ... x_{n+1}]$.
      \item[(ii)] \[
m[i]=
\begin{cases}
0, &\text{if $i \ge n$;}\\
min_{i\leq k\textless n+1} (m[k]+t[k]), &\text{if $i < n$;} 
\end{cases}
\]

Proof: \\
For $i\ge n$ is trivial. \\

For the other case we consider an optimal solution OPT for finding a solution on $<s_i,..,s_{n+1}>$ and let j be the index so OPT has the form $<s_j,..,s_{n+1}>$. Then $m[i]=$ cost in OPT for $< s_j,..,s_{n+1} >+ t(j) \ge m[j]+t(j) \ge \text{min}_{i\le l \le n+1}(m[l]+t(l))$ \\

For $i\le l \le n+1$ there is a solution of $m[l] + t(l)$ such that $m[i] \le \text(min)_{i \le l \le n+1}(m[l]+t(l))$.
      \item[(iii)] 
\begin{algorithmic}[1]
\STATE{CalcWaitTime(T)}
\FOR{$i \gets 1 to n$}
\STATE{m[i,i]$\gets 0$}
\ENDFOR
\FOR{$i \gets n-1 to 1$}
\STATE{m$[i] \gets $min$_{i\le k \le n+1}($m[$k$],T($k$))}
\ENDFOR
\RETURN{m[1]}
\end{algorithmic}

      \end{enumerate}
\item ($\frac{1}{2}+1+\frac{1}{2}+\frac{1}{2}+\frac{1}{2}$ point)
      Let $P=\langle p_1,\ldots,p_n \rangle$ be a sequence of $n$ points in the plane.
      No two points have the same $x$-coordinate, and the points are given from left to right.
      Any subsequence $P'$ of $P$ defines a path, which is given by visiting the points
      from $P'$ in order (that is, from left to right). The cost of this path, $\cost(P')$,
      is its total length. For example, if $P' = \langle p_2, p_4, p_5, p_8\rangle$ then
      $\cost(P') = |p_2p_4| + |p_4p_5| + |p_5p_8|$, where $|p_ip_j|$ denotes the
      length of the segment $p_ip_j$. We want to partition $P$ into two subsets $P_1$, $P_2$
      such that $\cost(P_1)+\cost(P_2)$ is minimized.
      \begin{enumerate}
      \item[(i)]

Given $i,j$ with $1 \le i < j \le n$ compute a partition for $[p_i,...,p_n]$ with minimal cost(P$_1$) + cost(P$_2$) where P$_1$ starts it’s curve at p$_i$ and P$_2$ start it’s curve at p$_j$.

      \item[(ii)] 
First we define the cost of a path, pCost(i,j)=$\sum_{k=i}^{j-1}|p_kp_{k+1}|$
\[
m[i]=
\begin{cases}
0, & $\text{if $i \ge n-1\wedge  j=n$;}$\\
min_{i < l \le n}(min_{l<k\le n}(m[l,k]+pCost(i,l-1)+|p_{l-1}p_k|)), & $\text{otherwise;}$
\end{cases}
\]
Proof:

If $i \ge n \wedge j=n$, this is a trivial case.

For the other case we consider a optimal solution OPT as a partition for [$p_1,..,p_n$] where cost($P_1$)+cost($P_2$) is minimized and let g be the index such that OPT is of the following form: [$p_g,..,p_n$].\\ Then $m[i,j]=$cost$(P_1)+$cost$(P_2)$ (for OPT) is greater or equal to $m[i,g]+$pCost$(i,g))$ is greater equal to $min_{i<l\le n}(min_{l<k\le n}(m[l,k]+pCost(i,l-1)+|p_{l-1}p_k|)))$\\

For any $i\le k < n $ we get that there is a solution: $m[l,k]+\text{pCost}(i,l-1)+|p_{l-1}p_k|$ such that there is an l so the following holds: $m[i,j]\le min_{i<l\le n}(min_{l<k\le n}(m[l,k]+pCost(i,l-1)+|p_{l-1}p_k))$.



      \item[(iii)] 
First lets compute the pCost
\begin{algorithmic}[1]
\STATE CalcPCost(i,j)
\FOR{$j \gets 1\text{ to }n$}
\STATE{pCost[$i,i+1]\gets |p_ip_{i+1}|$}
\FOR{$d \gets i + 2$}
\STATE{pCost[$i,j] \gets$ pCost[$i,j-1]+|p_{d-1}p_d|$}
\ENDFOR
\ENDFOR
\end{algorithmic}
\bigskip

\begin{algorithmic}[1]
\STATE{CalcPartition(P)}
\STATE{CalcPCost(i,j)}
\STATE{m[$n,n]\gets 0$}
\STATE{m[$n-1,n]\gets 0$}
\FOR{$j\gets n-1 \text{ to } 1$}
\FOR{$d \gets j+1 \text{ to } n$}
\STATE{m[$j,d]\gets$min$_{j<l\le n}($min$_{l<k\le n}($m[$l,k$]+pathCost[$i,l-1$]+$|p_{l-1}p_k|$))}
\ENDFOR
\ENDFOR
\RETURN{min$_{2\le g \le n}(m[1,g]$)}
\end{algorithmic}


      \item[(iv)] 
First lets start with CalcPCost, since we have 2 nested loops over $n$ while the rest of the function has contstant running time we get a total running time of $O(n^2)$.

CalcPartition on the other hand has 2 nested loops as well, but line 7 does not have constant running time. But since j is the determining value (because of $j<l\le k\le n$) this actually runs for $O(n)$ time.
Since line 7 runs in the 2 nested loops the actual running time is $O(n^3)$. Line 10 also is actually a loop - which runs for $O(n-2)$ - but since $O(n^3)>O(n-2)$ the running time of the algorithm is $O(n^3)$.
      \item[(v)] Extend the algorithm so that it not only computes the value of an optimal
                 partition, but the partition itself.
      \end{enumerate}
\end{rlist}

\end{document} 